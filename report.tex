% rm -f report.bib && pdflatex report && bibtex report && pdflatex report && pdflatex report
\documentclass[twocolumn]{article}

\IfFileExists{hyperref.sty}{\usepackage[pdfborder={0 0 0}]{hyperref}}{\newcommand{\url}{\verb}}

\title{Project 3: Training on Parameter Subspaces}
\author{Diego Bellani}
\date{2022}

\pagenumbering{gobble}

\begin{filecontents}[nosearch]{report.bib}
@inproceedings{
	intr,
	title={Measuring the Intrinsic Dimension of Objective Landscapes},
	author={Chunyuan Li and Heerad Farkhoor and Rosanne Liu and Jason Yosinski},
	booktitle={International Conference on Learning Representations},
	year={2018},
	url={https://openreview.net/forum?id=ryup8-WCW},
}
@misc{
	jgamper,
	title   = "Intrinsic-dimensionality Pytorch",
	author  = "Gamper, Jevgenij",
	year    = "2020",
	howpublished = "\url{https://github.com/jgamper/intrinsic-dimensionality}",
	url     = "https://github.com/jgamper/intrinsic-dimensionality"
}
@misc{
	mdda,
	title  = "Intrinsic Dimension: Proof of Concept",
	author = "Andrews, Martin",
	year   = "2018",
	howpublished = "\url{https://github.com/mdda/deep-learning-workshop/blob/master/notebooks/work-in-progress/IntrinsicDimension/}",
	url    = "https://github.com/mdda/deep-learning-workshop/blob/master/notebooks/work-in-progress/IntrinsicDimension/IntrinsicDimension.ipynb"
}
@article{
	mnist,
	title={The mnist database of handwritten digit images for machine learning research},
	author={Deng, Li},
	journal={IEEE Signal Processing Magazine},
	volume={29},
	number={6},
	pages={141--142},
	year={2012},
	publisher={IEEE}
}
@article{
	lenet,
	author={Lecun, Y. and Bottou, L. and Bengio, Y. and Haffner, P.},
	journal={Proceedings of the IEEE},
	title={Gradient-based learning applied to document recognition},
	year={1998},
	volume={86},
	number={11},
	pages={2278-2324}
}
@inproceedings{
	aghajanyan-etal-2021-intrinsic,
	title = "Intrinsic Dimensionality Explains the Effectiveness of Language Model Fine-Tuning",
	author = "Aghajanyan, Armen  and Gupta, Sonal  and Zettlemoyer, Luke",
	booktitle = "Proceedings of the 59th Annual Meeting of the Association for Computational Linguistics and the 11th International Joint Conference on Natural Language Processing (Volume 1: Long Papers)",
	month = aug,
	year = "2021",
	address = "Online",
	publisher = "Association for Computational Linguistics",
	url = "https://aclanthology.org/2021.acl-long.568",
	doi = "10.18653/v1/2021.acl-long.568",
	pages = "7319--7328",
}
@misc{
	blog1,
	title = "The Intrinsic Dimension of Objective Landscapes",
	author = "Carl Lemaire",
	year = 2018,
	month = may,
	address = "Online",
	howpublished ="\url{https://vitalab.github.io/article/2018/05/11/intrinsic-dimension-objective-landscapes.html}"
}
@misc{
	blog2,
	title = "Measuring the Intrinsic Dimension of Objective Landscapes (2018) - summary",
	author = "Tom Roth",
	year = 2020,
	month = jan,
	address = "Online",
	howpublished = "\url{https://tomroth.com.au/notes/intdim/intdim/}"
}
@inproceedings{
	fastfood,
	title = {Fastfood - Approximating Kernel Expansions in Loglinear Time},
	author = {Quoc Le and Tamas Sarlos and Alex Smola},
	year = {2013},
	booktitle = {30th International Conference on Machine Learning (ICML)}
}
@inproceedings{
	sparse,
	title = {Very sparse random projections},
	author = {Ping Li and Trevor J. Hastie and Kenneth Ward Church},
	booktitle = {Knowledge Discovery and Data Mining},
	year = {2006}
}
@article{
	lstm,
	author = {Hochreiter, Sepp and Schmidhuber, Jürgen},
	year = {1997},
	month = {12},
	pages = {1735-80},
	title = {Long Short-term Memory},
	volume = {9},
	journal = {Neural computation}
}
@article{
	trans,
	author = {Vaswani, Ashish and Shazeer, Noam and Parmar, Niki and Uszkoreit, Jakob and Jones, Llion and Gomez, Aidan and Kaiser, Lukasz and Polosukhin, Illia},
	year = {2017},
	month = {06},
	pages = {},
	title = {Attention Is All You Need}
}
@article{
	diffusion,
	title={Denoising Diffusion Probabilistic Models},
	author={Jonathan Ho and Ajay Jain and Pieter Abbeel},
	year={2020},
	journal={arXiv preprint arxiv:2006.11239}
}
@inproceedings{
	gan,
	title={Generative adversarial nets},
	author={Goodfellow, Ian and Pouget-Abadie, Jean and Mirza, Mehdi and Xu, Bing and Warde-Farley, David and Ozair, Sherjil and Courville, Aaron and Bengio, Yoshua},
	booktitle={Advances in neural information processing systems},
	pages={2672--2680},
	year={2014}
}
@misc{
	python,
	title = {Sunsetting Python 2},
	author = {Python Software Foundation},
	address = "Online",
	howpublished = "\url{https://www.python.org/doc/sunset-python-2/}"
}
\end{filecontents}

\begin{document}
\maketitle

\begin{abstract}
Li et al.~\cite{intr} proposed a way to determine how many parameters are really
needed to `solve' a task with a neural network by training it in a randomly
oriented subspace of their native parameter space. They call \emph{intrinsic
dimension of the objective (or loss) landscape}, the smaller dimension of this
subspace at which `solutions' first appear.

We started by reproducing part of the result that they showed. Than we reported
on the main properties of neural networks trained on a subspace of their
intrinsic dimension, compared them to standard smaller neural networks, tested
their performance for different subspace dimensions and tested two different
ways of projecting the randomly oriented subspace to the native parameter space.
\end{abstract}

\section{Introduction}
The number of trainable parameters required to solve a problem with a given
neural network can be intuitively used as a rough estimate of how difficult it
is to solve for that specific architecture. The intrinsic dimension of the
objective landscape, the authors claim, can help us find the minimum number of
trainable parameters needed for a task and then allow us to do some comparison
across different architectures and learning strategies like supervised and
reinforcement learning.

To give two concrete examples the authors found out that solving
MNIST~\cite{mnist} with LeNet~\cite{lenet} is approximately two times easier
than solving it with a fully-connected network and that playing Atari Pong from
pixels is 100 times harder than solving MNIST.

The author consider a problem `solved' if a trained model reaches a certain
accuracy. In the paper for MNIST the target accuracy is 90\% while for CIFAR-10
is just above 50\%.

The way in which we find the \emph{intrinsic dimension}\footnote{We are going to
use this shorter name to refer to the `intrinsic dimension of the objective
landscape'.} is relatively simple, we first generate a random projection matrix
$P_i$\footnote{Other kinds of projection matrices can be used here but for
simplicity we say a random one.} and an initial value $\theta_i^{(D_i)}$ for
each parameter in our network, were $D_i$ is the native dimension of that
parameter\footnote{Again for simplicity here we assume that all parameters of
the network (weights and biases) are vectors, because we can easily reshape
vectors in matrix form.}. Now let $D = \sum_i D_i$, then we generate a
\emph{single trainable parameter} $\theta^{(d)}$ for the entire network, where
$d<D$. At this point we can use the following formula
%
\begin{displaymath}
	\theta_i^{'(D_i)} =\theta_i^{(D_i)} + P_i \theta^{(d)}
\end{displaymath}
%
to offset the starting point of each parameter by a learned amount.

To find the intrinsic dimension we simply start to train the network with a
small $d$ and gradually increase it until solution starts to appear. By this the
authors mean that the network reaches 90\% of the accuracy of its normally
trained version, which they call the $d_{\mathrm{int}90}$ value.

\section{Related Works}
We did not do a systematic literature review for this project so here we are
just going to report what we have found with a shallow search on the topic. Also
we do not know of any other novel method of measuring the difficulty of a task
for a given architecture.

First we looked for other attempt at reproducing the results of the paper. We
only found two, a proof of concept by Andrews~\cite{mdda} and a failed attempt
by Gamper~\cite{jgamper}. So not a lot of serious attempts at reproduction were
made public.

A quick look at the literature revealed just another paper by Aghajanyan et
al.~\cite{aghajanyan-etal-2021-intrinsic} that used the method described by the
authors, even though the paper has some citations. Therefore, to a shallow look,
it does not seem that this metric is used in the academic circles.

Finally we found a few blog posts that tried to explain the content of the
article. Most of them were just public notes from other scholars like Lemaire
and Roth~\cite{blog1, blog2}.

\section{Method}
We started by reproducing a subset of the results in the original paper, some
problems encountered along the way are described in appendix \ref{app:troubles}.
We did not tackle the reinforcement learning results because they were out of
our competences.

Our work was focused on fully connected and convolutional neural networks
(LeNet5). The task were MNIST and CIFAR-10.

After having reproduced what was possible we proceeded to test the performance
of a smaller neural networks. To see if the intrinsic dimension found was really
the minimum numbers of parameter needed to do the given task with the chosen
architecture.

In the original article the authors test two other ways of projecting the
trainable parameter, namely the FastFood transform~\cite{fastfood} and the
sparse matrix projection~\cite{sparse}.

Due to hardware limitations reproducing the results for the FastFood transform
was not possible, therefore we only report on sparse matrix projections.

All the experiment we run on a 15-inch 2018 MacBook Pro with a 2,2 GHz Intel
Core i7 6 core with macOS Ventura 13.0.1. We used PyTorch \texttt{1.14.0.dev20221119}. All training and inference was done on the CPU because the
GPU backhand on macOS is still slower than the CPU.

\section{Results}
We report in table \ref{tab:params} the number of trainable parameters needed to
reach the same accuracy described in the paper, for the ``normal'' models, and
the $d_{\mathrm{int}90}$ value for both the randomly projected and small
architectures. We put an X near the results that we were not able to reproduce
due to hardware limitations.

The only intrinsic dimension we have found to be different from the one reported
in the paper is the one for IntrLeNet with the MNIST dataset, 300 instead of
290.

Using random projection did not allowed us to reproduce IntrLeNet with the MNIST
dataset either. Changing the projection used does not change the number of
trainable parameters, therefore is not reported in table~\ref{tab:params}.

\section{Conclusions}
Training networks on random subspaces is quite tricky, we had to get some
hyper-parameter just right to get it to learn and even when we got them right,
is some cases, we had to wait a few epochs before we were able to see some
improvements.

Even increasing the intrinsic dimension a bit does not makes things easier. Of
course the closer the intrinsic dimension gets to the original number of
parameters the easier the network becomes to train but it also gets slower and
slower.

Using dense random projection makes the training quite slow, but using sparse
random projections make the training really fast. The number of epochs required
goes from numbers in the thousands to no more than ten! This kind of projection
works so well that we strongly advise to use it in place of all the others. If
the reader needs to find the intrinsic dimension of a model sparse random
projection is definitely the way to go.

The method seems unable to find the smallest numbers of trainable parameters
needed to solve a problem for all architectures. We were able to archive the
same accuracy with a standard smaller neural network in the case of
convolutional networks. It may be useful to find a rough upper bound of the
numbers of parameters needed for a given network.

This method gives useful insights for fully-connected neural networks because it
gives us a number of parameters that is lower than the minimum possible i.e. the
intrinsic dimension for MNIST is 750 but just the input layer for that model has
$28 \cdot 28 = 784$ parameters. This may be an hint that there is a better way
to solve that problem. In the case of fully-connected networks the better method
is convolutional networks which does not look at the whole input at once. It may
tell us that there is still room for improvement for other models.

In conclusion we think that this method may be useful to see if the model used
to solve a given problem is `optimal' in the number of parameters or not. It
might be interesting for example to see for example what is the intrinsic
dimension of an LSTM~\cite{lstm} compared to a transformer~\cite{trans} for NLP
tasks or to compare diffusion models~\cite{diffusion} with GANs~\cite{gan} for
generative
tasks.

\begin{table}
	\centering
	\begin{tabular}{r|r|r|}
	\cline{2-3}
	                                 & MNIST   & CIFAR-10 \\
	\hline
	\multicolumn{1}{|r|}{FC}         & 199,210 & 656,810 \\
	\multicolumn{1}{|r|}{LeNet}      & 44,426  & 62,006  \\
	\multicolumn{1}{|r|}{IntrFC}     & 750     & X       \\
	\multicolumn{1}{|r|}{IntrLeNet}  & 300     & 2,900   \\
	\multicolumn{1}{|r|}{SmallFC}    & 1,600   & 46,495  \\
	\multicolumn{1}{|r|}{SmallLeNet} & 222     & 2,186   \\
	\hline
	\end{tabular}
	\caption{Numbers of trainable parameters needed by the various architectures
	to `solve' the two problems.}
	\label{tab:params}
\end{table}

\clearpage

\appendix

\section{Appendix}

\subsection{The pains of reproduction}\label{app:troubles}
As of today is hard to run their code due to the outdated version of Python and
Tensorflow used.  The authors implemented their ideas in Python~2.7, the paper
was published in 2018, by that point it was already known by 10 years (!) that
python 2.7 was going to be deprecated~\cite{python}. This shows no will in the
long term reproducibility of their code.

The paper is absolutely not clear about the fact that there should be only one
$\theta^{(d)}$ in the entire network. We got this fact wrong and slowed our
progresses in reproducing the results because using multiple trainable
$\theta^{(d)}$ also works in some cases!

To add insult to the injury their code is really convoluted, to say the least.
To multiply two tensors and sum another one, i.e. doing the projection, the
authors feel it is necessary to spread their code among multiple classes. This
makes it much more harder than it should to read their code and using it as an
eventual reference.

\bibliographystyle{plain}
\bibliography{report}

\end{document}
